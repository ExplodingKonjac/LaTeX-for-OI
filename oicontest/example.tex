\documentclass{oicontest}

\begin{document}

\title{NOI2023 模拟赛}
\author{}
\date{2023 年 6 月 10 日}
\noititle[带鱼,薯条,烧饼][fish,chips,pancake][3,1,2][1024,512,1024][5,5,4][否,否,否]

\newproblem{带鱼}{fish}

\background

这是出题人在吃(很长)的带鱼时想出来的题目。

\probdescription

\stress{为了降低输入输出压力,本题是一道交互题。}

有一个范围为 $[0,n]$ 的数轴,给定长度为 $n$ 的正整数序列 $a$(下标范围 $0\sim n-1$)以及非负整数 $K$。对于每一个整数 $i\in [0,n]$,解决如下问题:

选择一个集合 $S\subseteq [0,n-1]\cap\mathbb{Z}$,花费的代价为 $\sum_{x\in S}a_x$。设 $[l,r]\ (l,r\in\mathbb{Z})$ 为最长的区间满足 $i\in[l,r]$ 且 $\forall x\in S,x\notin [l,r-1]$。求出使得 $r-l\le K$ 所需要的最小代价。

我们下发了 \filename{fish.h} 和 \filename{grader.cpp} 用于测试你的程序,你可以用如下编译命令进行编译:

\begin{code}{bash}
g++ fish.cpp grader.cpp -o fish -std=c++14 -O2
\end{code}

\implement

\stress{你不需要,也不应该实现主函数。}

你需要包含头文件 \ilcode{cpp}{"fish.h"},并实现以下函数:

\begin{code}{cpp}
std::vector<int> minCost(int n,int K,std::vector<int> a)
\end{code}

其中,$n,K$ 的意义同题目描述,$a$ 是花费序列,保证长度为 $n$。交互库在每个测试点中会调用函数 \ilcode{cpp}{minCost} 恰好一次,你需要返回一个长度为 $n+1$ 的 \ilcode{cpp}{std::vector<int>},其中下标为 $i$ 的值表示 $i$ 的答案。

\stress{你不应该从标准输入读入任何东西,或者向标准输出输出任何东西,这些行为将被视为作弊。}

\subsection[交互库输入格式]{【交互库输入格式】}

这里是下发交互库的输入格式,用于帮助你测试你的程序。

第一行输入两个数表示 $n,K$。

第二行输入 $n$ 个整数表示序列 $a$。

\subsection[交互库输出格式]{【交互库输出格式】}

这里是下发交互库的输出格式,用于帮助你测试你的程序。

交互库会将你的输入作为测试数据,传入你实现的 \ilcode{cpp}{minCost} 函数中,并获取你的程序得到的结果。

交互库会输出 $n+1$ 行整数,第 $i$ 行表示你的程序得到的位置 $i$ 的结果。

\exampleinput{1}
\begin{example}
6 2
1 1 4 5 1 4
\end{example}

\exampleoutput{1}
\begin{example}
1
1
2
2
2
1
1
\end{example}

\examplefile{2}

该样例满足子任务 2 的限制。

\examplefile{3}

该样例满足子任务 4 的限制。

\constraints

对于 $100\%$ 的数据,$0\le K\le n\le 10^7,\ 1\le a_i\le 10^9$。

\begin{table}{c|c|c|c|c}
	子任务编号 & 子任务分值 & $n\le$ & $K\le$ & 特殊性质 \\
	\tabmid
	$1$ & $10$ & $100$ & $100$ & 无 \\
	\hline
	$2$ & $15$ & $10^3$ & $10^3$ & 无 \\
	\hline
	$3$ & $15$ & $10^5$ & $100$ & 无 \\
	\hline
	$4$ & $15$ & $10^6$ & $10^6$ & 保证 $a$ 单调不降 \\
	\hline
	$5$ & $45$ & $10^7$ & $10^7$ & 无 \\
\end{table}

\newproblem{薯条}{chips}

\background

这是出题人在吃(很软)的薯条时想出来的题目。

\probdescription

给定一个 $n$ 个点 $m$ 条边的无向图,保证 $m$ 为偶数,图联通且无重边、无自环。

对于图中度数为奇数的点,记有 $k$ 个。请构造方案将 $k$ 个点两两配对,并对每对点找出一条连接两点的包含偶数条边的路径,且每条边至多出现在一条路径中。

请输出方案或判断无解。

\inputformat

第一行两个正整数 $n,m$表示点数和边数。

接下来 $m$ 行,第 $i$ 行两个整数 $u_i,v_i$,表示无向边 $(u_i,v_i)$。
	
\outputformat

若无解,需要输出一行 \ttstring{-1}。否则,依次输出 $\frac{k}{2}$ 对的方案。

对于第 $i$ 对,第一行输出三个整数 $a_i,b_i,l_i$,表示第 $i$ 对结点是 $(a_i,b_i)$,路径长为 $l_i$。

第二行输出 $l_i$ 个整数,表示路径上从 $a_i$ 到 $b_i$ 顺次经过的边的编号。

如果有多种方案,你可以输出任意一种。

\exampleinput{1}
\begin{example}
5 8
1 2
2 3
3 4
4 1
1 5
2 5
3 5
4 5
\end{example}

\exampleoutput{1}
\begin{example}
1 4 2
5 8
2 3 6
2 7 6 1 4 3
\end{example}

\exampleinput{2}
\begin{example}
8 8
1 2
2 3
3 4
2 5
2 6
3 6
3 7
3 8
\end{example}

\exampleoutput{2}
\begin{example}
1 5 2
1 4
3 4 4
6 5 2 3
7 8 2
7 8
\end{example}

\examplefile{3}

\examplefile{4}

\constraints

对于所有数据,满足 $2 \le n,m \le 3 \times 10^5 , 2 \le k \le n $,图联通且无重边、无自环。

\begin{table}{c|c|c}
	子任务编号 & 子任务分值 & 特殊限制\\
	\tabmid
	$1$ & $21$ & $n,m \le 20$\\
	\hline
	$2$ & $12$ & $k=2$\\
	\hline
	$3$ & $17$ & $m=n-1$\\
	\hline
	$4$ & $23$ & $n,m \le 3000$\\
	\hline
	$5$ & $10$ & 无\\
	\hline
\end{table}

\newproblem{烧饼}{pancake}

\background

这是出题人在吃(很圆)的烧饼时想出来的题目。

\probdescription

给定平面上 $n$ 个点 $P_1,P_2,\dots, P_n$,以及 $q$ 次查询。每次查询给定 $l,r$,求 $\{P_l,P_{l+1},\dots,P_r\}$ 的最小圆覆盖的半径。

\inputformat

第一行,两个正整数 $n, q$。

之后 $n$ 行,第 $i$ 行两个正整数 $x_i,y_i$,表示 $P_i$ 的坐标。

之后 $q$ 行,每行两个正整数 $l,r$,表示一次询问。

\outputformat

输出 $q$ 行,第 $i$ 行输出一个小数表示第 $i$ 次询问的答案。

对每组询问,设标准答案为 $\mathrm{ans}$ 而你的答案为 $\mathrm{out}$,则你的答案被认为是正确的当且仅当 $\frac{|\mathrm{out}-\mathrm{ans}|}{\max(1,\mathrm{ans})}\le 10^{-4}$。你的输出正确当且仅当 $q$ 次询问的答案均正确。

\exampleinput{1}
\begin{example}
10 10
665807 810957
853840 435163
437691 228975
831731 596470
792430 516054
620620 353584
823622 141842
709435 705175
600288 142903
224567 464849
3 5
1 6
5 9
2 8
7 9
4 5
4 7
7 10
3 6
5 9
\end{example}

\exampleoutput{1}
\begin{example}
269406.800501120
312548.384328856
292492.908126728
300508.141400720
292492.908126728
44752.937492974
227350.156292557
352145.534406940
269406.800501120
292492.908126728
\end{example}

\constraints

对于所有数据,$n\le 5\times 10^5,\ q\le 5000,\ x_i,y_i\le 10^9$。

\begin{table}{c|c|c|c}
	子任务编号 & 子任务分值 & $n\le$ & $q\le$ \\
	\tabmid
	$1$ & $20$ & $100$ & $100$ \\
	\hline
	$2$ & $20$ & $5\times 10^5$ & $1$ \\
	\hline
	$3$ & $30$ & $2.5\times 10^5$ & $2500$ \\
	\hline
	$4$ & $30$ & $5\times 10^5$ & $5000$ \\
\end{table}

\end{document}